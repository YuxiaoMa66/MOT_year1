% \special{dvipdfmx:config z 0}
\documentclass[UTF8,a4paper,AutoFakeBold,AutoFakeSlant]{article}
\usepackage[a4paper,left=2.8cm,right=2.6cm,top=3.7cm,bottom=3.5cm]{geometry}
\usepackage{ctex}
% \usepackage{xeCJK}
\usepackage{graphicx}
\usepackage{pythonhighlight}
\usepackage[mathscr]{eucal}
\usepackage{mathrsfs}
\usepackage{booktabs}
\usepackage{capt-of} 
\usepackage{hyperref} 
\usepackage{abstract}
\usepackage{amsmath}
\usepackage{listings}
\usepackage{color}
\usepackage{caption}
\usepackage{subfigure}
\usepackage{enumerate}
\usepackage{amsfonts} 
\usepackage{CJK,CJKnumb}
\usepackage{float}
% \usepackage{gbt7714}
\usepackage{framed}
\usepackage{multirow}
\usepackage{animate}
\usepackage[framemethod=tikz]{mdframed}


\newcommand{\tnewroman}{\fontspec{Times New Roman}}
\newcommand{\song}{\CJKfamily{song}}    % 宋体   (Windows自带simsun.ttf)
\newcommand{\fs}{\CJKfamily{fs}}        % 仿宋体 (Windows自带simfs.ttf)
\newcommand{\kai}{\CJKfamily{kai}}      % 楷体   (Windows自带simkai.ttf)
\newcommand{\hei}{\CJKfamily{hei}}      % 黑体   (Windows自带simhei.ttf)
\newcommand{\li}{\CJKfamily{li}}        % 隶书   (Windows自带simli.ttf) 
\newcommand{\ssong}{\CJKfamily{STSong}}
% \newfontfamily{\heiti}{SimHei}

\xeCJKsetup{SlantFactor = 0.3}
% \xeCJKsetup{SlantFactor = -0.7}
\setCJKmainfont[BoldFont=SimHei, SlantedFont=KaiTi]{SimSun}



\usepackage{xcolor}  	%高亮使用的颜色
\definecolor{commentcolor}{RGB}{85,139,78}
\definecolor{stringcolor}{RGB}{206,145,108}
\definecolor{keywordcolor}{RGB}{34,34,250}
\definecolor{backcolor}{RGB}{220,220,220}

\usepackage{accsupp}	
\newcommand{\emptyaccsupp}[1]{\BeginAccSupp{ActualText={}}#1\EndAccSupp{}}






\title{\textbf{\textsf{{\textsf{Pre-lecture assignment 2}}}}} 
\author{\tnewroman Yuxiao Ma~~~~~5916305}
\date{}

% 去掉红框
\hypersetup{
colorlinks=true,
linkcolor=black
}

\begin{document}



\maketitle


\section{Argurements}

\subsection{}

The validity is if $p \rightarrow q$ is valid, then if the p is true, the q must be true; 
or if the p is false, the truth value of q does not matter whether it is true or false. This condition is always valid.


\subsection{}

A valid argument is one in which the conclusion necessarily follows from the premises. In other words, if all of the premises are true, then the conclusion must also be true.
A valid argument can have a false conclusion if its premises are false.

A sound argument is a valid argument in which all of the premises are true. A sound argument is a stronger requirement than validity, as it requires that the premises be true in addition to the argument being valid.
A sound argument cannot have a false conclusion.


\subsection{}

Aruguement themselves can neither be true nor false.
An argument is valid if and only if the truth of its premises entails the truth of its conclusion.


\subsection{}

P1: All fish live in water.

P2: Whales live in water.

C: Whales are fish.

The argument is valid, but the conclusion is false.


\subsection{}

No, the argument is not valid. The conclusion (2+2=4) does not necessarily follow from the premises.


\subsection{}

P1: Light is a particle.

P2: Light is a wave.

C: Light is a particle and at the same time a wave.



\section{Propositional Logic}

\subsection{}

$ P \lor \lnot P $


\subsection{}

% draw truth table of $ P \lor \lnot P $
The truth table is as follows:
\begin{table}[H]
\centering
\begin{tabular}{|c|c|}
\hline
$P$ & $P \lor \lnot P$ \\ \hline
T   & T                \\ \hline
F   & T                \\ \hline
\end{tabular}
\end{table}


\subsection{}

\subsubsection{}
$A \rightarrow B$

\subsubsection{}
% draw truth table of $ A \rightarrow B $
The truth table is as follows:
\begin{table}[H]
\centering
\begin{tabular}{|c|c|c|}
\hline
$A$ & $B$ & $A \rightarrow B$ \\ \hline
T   & T   & T                 \\ \hline
T   & F   & F                 \\ \hline
F   & T   & T                 \\ \hline
F   & F   & T                 \\ \hline
\end{tabular}
\end{table}

\subsubsection{}
To demonstrate that the argument is valid, we need to show that the conclusion (C) is always true whenever the premises (P1 and P2) are both true. 
Looking at the truth table, we can see that the only row where both P1 and P2 are true is the first row. In this row, C is also true. Therefore, we can conclude that the argument is valid.


\subsection{}

\subsubsection{}
Rewrite the argument in symbolic form:
\begin{itemize}
    \item R: It rains
    \item W: The floor is wet
\end{itemize}
Therefore, the argument can be rewritten as:
\begin{itemize}
    \item P1: $R \rightarrow W$
    \item P2: $W$
    \item C: $R$
\end{itemize}

\subsubsection{}
% draw truth table of $ P1 \land P2 \rightarrow C $
The truth table is as follows:
\begin{table}[H]
\centering
\begin{tabular}{|c|c|c|c|}
\hline
$R$ & $W$ & $R \rightarrow W$ & Conclusion $R$ \\ \hline
T   & T   & T                 & T              \\ \hline
T   & F   & F                 & T              \\ \hline
F   & T   & T                 & F              \\ \hline
F   & F   & T                 & F              \\ \hline
\end{tabular}
\end{table}

\subsubsection{}
An argument is considered valid if the conclusion is always true whenever the premises are both true.
And the premises are $R \rightarrow W$ and $W$, the conclusion is $R$.
The only scenario where both premises are true is on the first line where $R$ and $W$ are both true.
However, in the third line the premises are true but the conclusion is false.(The floor is wet but it does not rain.)

Therefore, the argument is invalid.

\begin{itemize}
    \item bonus question:
\end{itemize}

"Affirming the consequent," and its Latin name is "Affirmatio consequentis."


\subsection{}
% draw truth table of $ P, Q, P\rightarrow Q, P \leftrightarrow Q $
The truth table is as follows:
\begin{table}[H]
\centering
\begin{tabular}{|c|c|c|c|c|}
\hline
$P$ & $Q$ & $P \rightarrow Q$ & $Q \rightarrow P$ &$P \leftrightarrow Q$($P \rightarrow Q \land Q \rightarrow P$)  \\ \hline
T   & T   & T                 & T &T                       \\ \hline
T   & F   & F                 & T &F                       \\ \hline
F   & T   & T                 & F &F                       \\ \hline
F   & F   & T                 & T &T                       \\ \hline
\end{tabular}
\end{table}

Observe that the columns for \( P \rightarrow Q \) and \( Q \rightarrow P \) must both be true 
for \( P \leftrightarrow Q \) to be true. Conversely, if either \( P \rightarrow Q \) or 
\( Q \rightarrow P \) is false, then \( P \leftrightarrow Q \) is also false.

% The last column shows the result of the 
% biconditional \( p \leftrightarrow q \). 
% If you compare this with the logical AND of \( P \rightarrow Q \) and \( Q \rightarrow P \), you'll notice that they are identical in all cases. Therefore, the statement \( (p \rightarrow q) \) and \( (q \rightarrow p) \) is logically equivalent to \( p \leftrightarrow q \), as demonstrated by the matching values in the last column of the truth table. This proves the equivalence of the two expressions using a truth table.


\subsection{}

Premise 1: All dogs are mammals. (P1: D $\rightarrow$ M)

Premise 2: Samoyed is a dog. (P2: D)

Conclusion: Therefore, Samoyed is a mammal. (C: M)

\begin{table}[H]
    \centering
    \begin{tabular}{|c|c|c|c|c|}
    \hline
    $D$ & $M$ & $P_1$ & $P_2$ & $C$ \\ \hline
    T & T & T & T & T \\ \hline
    T & F & F & T & F \\ \hline
    F & T & T & T & F \\ \hline
    F & F & T & T & F \\ \hline
    \end{tabular}
\end{table}


\subsection{}

Premise 1: All birds have feathers. (P1: B $\rightarrow$ F)

Premise 2: Ostriches are birds. (P2: B)

Conclusion: Therefore, ostriches have feathers. (C: F)

\begin{table}[ht]
    \centering
    \begin{tabular}{|c|c|c|c|c|}
    \hline
    $B$ & $F$ & $P_1$ & $P_2$ & $C$ \\ \hline
    T & T & T & T & T \\ \hline
    T & F & F & T & F \\ \hline
    F & T & T & T & F \\ \hline
    F & F & T & T & T \\ \hline
    \end{tabular}
\end{table}
    


\section{Propositional Logic (difficult)}

\subsection{}

Define:
\begin{itemize}
    \item B(x) to be the predicate "x is a ball"
    \item R(x) to be the predicate "x is red"
\end{itemize}

Translate "the ball is red" into predicate logic as follows:
\begin{equation}
    \exists x(B(x) \land R(x))
\end{equation} 
This states that b is a ball and b is red.


\subsection{}

The sentence "all balls are red" can be translated into predicate logic as:
\begin{equation}
    \forall x(B(x) \land R(x))
\end{equation}
For all x, if x is a ball, then x is red. Every object that is a ball is also red. This is equivalent to saying that "all balls are red.


\subsection{}
\begin{table}[H]
\begin{tabular}{cc|c|c|c}
    \hline
    \( P \) & \( Q \) & \( P \rightarrow Q \) & \( \neg (P \wedge (Q \rightarrow P) \wedge (P \rightarrow \neg Q)) \) & \( ((P \rightarrow Q) \vee \neg (P \wedge (Q \rightarrow P) \wedge (P \rightarrow \neg Q))) \) \\
    \hline
    T & T & T & T & T \\
    T & F & F & F & F \\
    F & T & T & T & T \\
    F & F & T & T & T \\
    \hline
    \end{tabular} 
\end{table}   


% \bibliographystyle{gbt7714-numerical}
% % \bibliographystyle{7714-author-year}
% \bibliographystyle{ieeetr}
% \bibliography{bibl}

\end{document}


% \bibliographystyle{gbt7714-numerical}
% % \bibliographystyle{7714-author-year}
% \bibliographystyle{ieeetr}
% \bib