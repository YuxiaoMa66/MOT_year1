% \special{dvipdfmx:config z 0}
\documentclass[UTF8,a4paper,AutoFakeBold,AutoFakeSlant]{article}
\usepackage[a4paper,left=2.8cm,right=2.6cm,top=3.7cm,bottom=3.5cm]{geometry}
\usepackage{ctex}
% \usepackage{xeCJK}
\usepackage{graphicx}
\usepackage{pythonhighlight}
\usepackage[mathscr]{eucal}
\usepackage{mathrsfs}
\usepackage{booktabs}
\usepackage{capt-of} 
\usepackage{hyperref} 
\usepackage{abstract}
\usepackage{amsmath}
\usepackage{listings}
\usepackage{color}
\usepackage{caption}
\usepackage{subfigure}
\usepackage{enumerate}
\usepackage{amsfonts} 
\usepackage{CJK,CJKnumb}
\usepackage{float}
% \usepackage{gbt7714}
\usepackage{framed}
\usepackage{multirow}
\usepackage{animate}
\usepackage[framemethod=tikz]{mdframed}

\newcommand{\tnewroman}{\fontspec{Times New Roman}}
\newcommand{\song}{\CJKfamily{song}}    % 宋体   (Windows自带simsun.ttf)
\newcommand{\fs}{\CJKfamily{fs}}        % 仿宋体 (Windows自带simfs.ttf)
\newcommand{\kai}{\CJKfamily{kai}}      % 楷体   (Windows自带simkai.ttf)
\newcommand{\hei}{\CJKfamily{hei}}      % 黑体   (Windows自带simhei.ttf)
\newcommand{\li}{\CJKfamily{li}}        % 隶书   (Windows自带simli.ttf) 
\newcommand{\ssong}{\CJKfamily{STSong}}
% \newfontfamily{\heiti}{SimHei}

\xeCJKsetup{SlantFactor = 0.3}
% \xeCJKsetup{SlantFactor = -0.7}
\setCJKmainfont[BoldFont=SimHei, SlantedFont=KaiTi]{SimSun}



\usepackage{xcolor}  	%高亮使用的颜色
\definecolor{commentcolor}{RGB}{85,139,78}
\definecolor{stringcolor}{RGB}{206,145,108}
\definecolor{keywordcolor}{RGB}{34,34,250}
\definecolor{backcolor}{RGB}{220,220,220}

\usepackage{accsupp}	
\newcommand{\emptyaccsupp}[1]{\BeginAccSupp{ActualText={}}#1\EndAccSupp{}}






\title{\textbf{\textsf{{\textsf{Reflection 1}}}}} 
\author{\tnewroman Yuxiao Ma~~~~~5916305}
\date{}

% 去掉红框
\hypersetup{
colorlinks=true,
linkcolor=black
}

\begin{document}



\maketitle


\begin{quotation}
	Science doesn’t need philosophy, anymore.
\end{quotation}

I disagree with this statement. Science and philosophy are deeply connected 
and this relationship is particularly evident in the field of epistemology.
Epistemology is the study of knowledge and justified belief. Philosophy provides
a framework for understanding the nature of knowledge and belief, which is critical and can be applied to the scientific method.
The scientific method is a process of investigation in which 
problems are identified and their explanations are tested by carrying out experiments.
The scientific method is a way of knowing about the world. And philosophy helps clarify the concepts, 
define terms and formulate hypotheses. 

Moreover, philosophy is also important in the field of ethics, which is vital to guide scientific practice.
For example, philosophical discussions about the nature of identity and human rights play a significant
role in the debate about the ethics of human cloning and even the recent development of AI technology.

Furthermore, philosophy research can challenge the assumptions of science and help scientists to think outside the box, 
ensuring that science does not operate in an "echo chamber". This interation between science and philosophy is crucial 
in preventing science from becoming dogmatic and unable to progress.



% \bibliographystyle{gbt7714-numerical}
% % \bibliographystyle{7714-author-year}
% \bibliographystyle{ieeetr}
% \bibliography{bibl}

\end{document}