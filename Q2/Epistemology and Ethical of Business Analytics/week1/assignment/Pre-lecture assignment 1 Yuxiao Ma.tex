% \special{dvipdfmx:config z 0}
\documentclass[UTF8,a4paper,AutoFakeBold,AutoFakeSlant]{article}
\usepackage[a4paper,left=2.8cm,right=2.6cm,top=3.7cm,bottom=3.5cm]{geometry}
\usepackage{ctex}
% \usepackage{xeCJK}
\usepackage{graphicx}
\usepackage{pythonhighlight}
\usepackage[mathscr]{eucal}
\usepackage{mathrsfs}
\usepackage{booktabs}
\usepackage{capt-of} 
\usepackage{hyperref} 
\usepackage{abstract}
\usepackage{amsmath}
\usepackage{listings}
\usepackage{color}
\usepackage{caption}
\usepackage{subfigure}
\usepackage{enumerate}
\usepackage{amsfonts} 
\usepackage{CJK,CJKnumb}
\usepackage{float}
% \usepackage{gbt7714}
\usepackage{framed}
\usepackage{multirow}
\usepackage{animate}
\usepackage[framemethod=tikz]{mdframed}

\newcommand{\tnewroman}{\fontspec{Times New Roman}}
\newcommand{\song}{\CJKfamily{song}}    % 宋体   (Windows自带simsun.ttf)
\newcommand{\fs}{\CJKfamily{fs}}        % 仿宋体 (Windows自带simfs.ttf)
\newcommand{\kai}{\CJKfamily{kai}}      % 楷体   (Windows自带simkai.ttf)
\newcommand{\hei}{\CJKfamily{hei}}      % 黑体   (Windows自带simhei.ttf)
\newcommand{\li}{\CJKfamily{li}}        % 隶书   (Windows自带simli.ttf) 
\newcommand{\ssong}{\CJKfamily{STSong}}
% \newfontfamily{\heiti}{SimHei}

\xeCJKsetup{SlantFactor = 0.3}
% \xeCJKsetup{SlantFactor = -0.7}
\setCJKmainfont[BoldFont=SimHei, SlantedFont=KaiTi]{SimSun}



\usepackage{xcolor}  	%高亮使用的颜色
\definecolor{commentcolor}{RGB}{85,139,78}
\definecolor{stringcolor}{RGB}{206,145,108}
\definecolor{keywordcolor}{RGB}{34,34,250}
\definecolor{backcolor}{RGB}{220,220,220}

\usepackage{accsupp}	
\newcommand{\emptyaccsupp}[1]{\BeginAccSupp{ActualText={}}#1\EndAccSupp{}}






\title{\textbf{\textsf{{\textsf{Pre-lecture assignment 1}}}}} 
\author{\tnewroman Yuxiao Ma~~~~~5916305}
\date{}

% 去掉红框
\hypersetup{
colorlinks=true,
linkcolor=black
}

\begin{document}



\maketitle


\section{}

\begin{itemize}
	\item Descriptive:
\end{itemize}

b, e, h, i, j, l, m, n

\begin{itemize}
	\item Normative:
\end{itemize}

a, c, d, f, g, k


\section{}

\begin{itemize}
	\item Inductive reasoning:
\end{itemize}

b, c

\begin{itemize}
	\item Deductive reasoning:
\end{itemize}

a, d


\section{}

No. Because cunducting science without philosophical presuppositions is vitually impossible.
The presuppositions are the foundation of science, which is steeply rooted in the philosophy of science.
Science and philosophy are deeply connected 
and this relationship is particularly evident in the field of epistemology.
Epistemology is the study of knowledge and justified belief. Philosophy provides
a framework for understanding the nature of knowledge and belief, which is critical and can be applied to the scientific method.

The princlples of logioc and reasoning, essential tools in formaulating and testing hypotheses, are also philosophical in nature.
This indicates that philosophical arguments guide scientists on how to conduct science, in which the presuppositions play a significant role.
Without these presuppositions, science would lack direction and purpose, and would be unable to progress.

Additionally, Scientists must assume that there is an external world that can be understood, and 
that there are universal laws that can be discovered. These are not empirical claims that can be tested; 
they are philosophical presuppositions that form the bedrock of scientific investigation.

Furthermore, the problem of determining what is considered science and what is not is a philosophical issue. 
The criteria that scientists use to judge whether a theory is good, such as whether it can be disproved, 
whether it makes accurate predictions, and whether it is simple, are philosophical ideas.
Thus, science cannot be conducted without philosophical presuppositions.



% \bibliographystyle{gbt7714-numerical}
% % \bibliographystyle{7714-author-year}
% \bibliographystyle{ieeetr}
% \bibliography{bibl}

\end{document}