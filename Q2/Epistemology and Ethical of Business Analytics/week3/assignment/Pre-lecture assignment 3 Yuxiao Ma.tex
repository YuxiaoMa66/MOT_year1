% \special{dvipdfmx:config z 0}
\documentclass[UTF8,a4paper,AutoFakeBold,AutoFakeSlant]{article}
\usepackage[a4paper,left=2.8cm,right=2.6cm,top=3.7cm,bottom=3.5cm]{geometry}
\usepackage{ctex}
% \usepackage{xeCJK}
\usepackage{graphicx}
\usepackage{pythonhighlight}
\usepackage[mathscr]{eucal}
\usepackage{mathrsfs}
\usepackage{booktabs}
\usepackage{capt-of} 
\usepackage{hyperref} 
\usepackage{abstract}
\usepackage{amsmath}
\usepackage{listings}
\usepackage{color}
\usepackage{caption}
\usepackage{subfigure}
\usepackage{enumerate}
\usepackage{amsfonts} 
\usepackage{CJK,CJKnumb}
\usepackage{float}
% \usepackage{gbt7714}
\usepackage{framed}
\usepackage{multirow}
\usepackage{animate}
\usepackage[framemethod=tikz]{mdframed}


\newcommand{\tnewroman}{\fontspec{Times New Roman}}
\newcommand{\song}{\CJKfamily{song}}    % 宋体   (Windows自带simsun.ttf)
\newcommand{\fs}{\CJKfamily{fs}}        % 仿宋体 (Windows自带simfs.ttf)
\newcommand{\kai}{\CJKfamily{kai}}      % 楷体   (Windows自带simkai.ttf)
\newcommand{\hei}{\CJKfamily{hei}}      % 黑体   (Windows自带simhei.ttf)
\newcommand{\li}{\CJKfamily{li}}        % 隶书   (Windows自带simli.ttf) 
\newcommand{\ssong}{\CJKfamily{STSong}}
% \newfontfamily{\heiti}{SimHei}

\xeCJKsetup{SlantFactor = 0.3}
% \xeCJKsetup{SlantFactor = -0.7}
\setCJKmainfont[BoldFont=SimHei, SlantedFont=KaiTi]{SimSun}



\usepackage{xcolor}  	%高亮使用的颜色
\definecolor{commentcolor}{RGB}{85,139,78}
\definecolor{stringcolor}{RGB}{206,145,108}
\definecolor{keywordcolor}{RGB}{34,34,250}
\definecolor{backcolor}{RGB}{220,220,220}

\usepackage{accsupp}	
\newcommand{\emptyaccsupp}[1]{\BeginAccSupp{ActualText={}}#1\EndAccSupp{}}






\title{\textbf{\textsf{{\textsf{Pre-lecture assignment 3}}}}} 
\author{\tnewroman Yuxiao Ma~~~~~5916305}
\date{}

% 去掉红框
\hypersetup{
colorlinks=true,
linkcolor=black
}

\begin{document}



\maketitle


\section{Statements falsifiable or not}

\begin{enumerate}
    \item Falsifiable: If we could find a chemical reaction that does not conserve mass, then this statement would be false.
    \item Not Falsifiable: In its current form, the statement is not falsifiable. Because if we define extra-terrestrial life as life that is different from earthly life, that is, it has no cell structure, no metabolism and inheritance, no carbohydrates and other characteristics, then it will be difficult for us to find and identify evidence of extra-terrestrial life. Because we do not have a universal standard and method to judge what is life and what is not life, we may ignore or misunderstand some unconventional life forms, such as silicon-based life, virtual life, etc. These life forms may exist in places that we cannot observe and detect. Alternatively, these life forms may exist in places we can observe and detect, but we cannot identify and understand them. In this case, we cannot prove or disprove the existence of extra-terrestrial life.
    \item Not Falsifiable: It is more a theoretical economic principle than a testable hypothesis. It's difficult to empirically test or disprove because it's broadly defined and "in the long run" is not a specific, measurable timeframe.
    \item Falsifiable: Can be tested by measuring angles.
    \item Not Falsifiable: Vague term "best adapted"
    \item Falsifiable: Can be tested against the established laws of physics.
    \item Falsifiable: Can be tested. It refers to a particular particle (could be identified and observed), and its behavior can be measured and potentially disproven.
\end{enumerate}


\section{Explain the Quine-Duhem thesis}

The Quine-Duhem thesis is that it is impossible to test a scientific hypothesis in isolation because an empirical 
test of the hypothesis requires one or more background assumptions.
It challenges the straightforward interpretation of scientific experiments and the testing of hypotheses.
The Quine-Duhem thesis tells that when we conduct an experiment to test a particular hypothesis within a scientific theory,
we are testing a complex web of interconnected hypotheses and assumptions that make uo the entire theoretical framwork.

In this thesis, it implies that if an experimental result contradicts our expections, it's not clear which part of this 
theoretical network is responsible for the contradiction. It could be the hypothesis we are testing, or it could be one of the
background assumptions which may be the broader theory. For example, there could be errors in the assumptions abnout the 
measurement tools, or there could be errors in the assumptions about the experimental setup.

Therefore, it is challenging to outright reject or falsify a scientific theory based on a single experiment.
This perspective is in contrast to the traditional view of
falsificationism, which states that a single experiment can be used to falsify a hypothesis (Karl Popper).
It suggests that scientific knowledge is a dynamic and interconnected system, where changing one part can have 
implcations for the whole system. It also suggests that scientific knowledge is not absolute, but rather, it is
subject to change and revision as new evidence and information becomes available.






% \bibliographystyle{gbt7714-numerical}
% % \bibliographystyle{7714-author-year}
% \bibliographystyle{ieeetr}
% \bibliography{bibl}

\end{document}


