% \special{dvipdfmx:config z 0}
\documentclass[UTF8,a4paper,AutoFakeBold,AutoFakeSlant]{article}
\usepackage[a4paper,left=2.8cm,right=2.6cm,top=3.7cm,bottom=3.5cm]{geometry}
\usepackage{ctex}
% \usepackage{xeCJK}
\usepackage{graphicx}
\usepackage{pythonhighlight}
\usepackage[mathscr]{eucal}
\usepackage{mathrsfs}
\usepackage{booktabs}
\usepackage{capt-of} 
\usepackage{hyperref} 
\usepackage{abstract}
\usepackage{amsmath}
\usepackage{listings}
\usepackage{color}
\usepackage{caption}
\usepackage{subfigure}
\usepackage{enumerate}
\usepackage{amsfonts} 
\usepackage{CJK,CJKnumb}
\usepackage{float}
% \usepackage{gbt7714}
\usepackage{framed}
\usepackage{multirow}
\usepackage{animate}
\usepackage[framemethod=tikz]{mdframed}


\newcommand{\tnewroman}{\fontspec{Times New Roman}}
\newcommand{\song}{\CJKfamily{song}}    % 宋体   (Windows自带simsun.ttf)
\newcommand{\fs}{\CJKfamily{fs}}        % 仿宋体 (Windows自带simfs.ttf)
\newcommand{\kai}{\CJKfamily{kai}}      % 楷体   (Windows自带simkai.ttf)
\newcommand{\hei}{\CJKfamily{hei}}      % 黑体   (Windows自带simhei.ttf)
\newcommand{\li}{\CJKfamily{li}}        % 隶书   (Windows自带simli.ttf) 
\newcommand{\ssong}{\CJKfamily{STSong}}
% \newfontfamily{\heiti}{SimHei}

\xeCJKsetup{SlantFactor = 0.3}
% \xeCJKsetup{SlantFactor = -0.7}
\setCJKmainfont[BoldFont=SimHei, SlantedFont=KaiTi]{SimSun}



\usepackage{xcolor}  	%高亮使用的颜色
\definecolor{commentcolor}{RGB}{85,139,78}
\definecolor{stringcolor}{RGB}{206,145,108}
\definecolor{keywordcolor}{RGB}{34,34,250}
\definecolor{backcolor}{RGB}{220,220,220}

\usepackage{accsupp}	
\newcommand{\emptyaccsupp}[1]{\BeginAccSupp{ActualText={}}#1\EndAccSupp{}}






\title{\textbf{\textsf{{\textsf{Pre-lecture assignment 4}}}}} 
\author{\tnewroman Yuxiao Ma~~~~~5916305}
\date{}

% 去掉红框
\hypersetup{
colorlinks=true,
linkcolor=black
}

\begin{document}



\maketitle


\section{Answer to question}

\begin{enumerate}
    \item A scientific theory is said to be predictively successful just in case it makes predictions that are borne out by observation and experimentation.
    \item Scientific realists hold an epistemically positive attitude towards aspects of scientific theories that they deem worthy of belief. This includes various forms such as Explanationist Realism, Entity Realism, Structural Realism, and Relative Realism.
    \item Anti-realists, like instrumentalists, don't regard scientific theories as literally true or false, but rather as useful or not. They see theories as instruments for practical goals like predicting natural phenomena, rather than attempts to describe reality's underlying nature.
    \item For constructive empiricists, a theory is empirically adequate if what it says about observable phenomena is true. This differs from a theory being literally true, which would encompass both observable and unobservable aspects of reality.
    \item Structural realists focus their realist commitment on unobservable structures, such as those represented in the equations of scientific theories, rather than on unobservable entities, processes, or events.
    \item The core disagreement between scientific realists and anti-realists revolves around the semantic stance of scientific realism. Realists believe scientific theories should be taken literally, meaning they can be either true or false, while anti-realists have a more skeptical view of this.
    \item The varieties of realism, such as Explanationist Realism, Entity Realism, Structural Realism, and Relative Realism, are distinguished by different views on the nature of reality, the role of experience, and the objectivity of knowledge. 
\end{enumerate}



\section{Additional questions}

\begin{enumerate}
    \item Main Argument for Realism: The key argument supporting realism, known as the “miracle” argument or the “no miracles” argument, posits that the predictive success and explanatory power of scientific theories suggest they are likely true or approximately true.

    Main Argument for Anti-Realism: Anti-realists argue that the focus should be on the predictive success of scientific theories rather than their truth or approximate truth. They view scientific theories as tools for practical goals like predicting natural phenomena, not as descriptions of reality's underlying nature.
    \item Difference Between Constructive Empiricism and Instrumentalism: Constructive empiricism holds that the aim of science is empirical adequacy, meaning we should believe what scientific theories say about observable phenomena but not about unobservables. In contrast, instrumentalism views scientific theories not as true or false, but as mere instruments for attaining practical goals like predicting natural phenomena.
    \item Machine Learning Algorithms' Support for Realism or Anti-Realism: Based on the definitions of these positions, machine learning algorithms, due to their underlying structure to the world that can be captured by mathematical models, could adds support to realism. However, since machine learning algorithms are not able to explain the underlying nature of reality, they could also be used to support anti-realism as they show that they are useful tools for practical goals like predicting natural phenomena without fully understanding the underlying causal mechanisms.
    \item Difference Between Epistemic Structural Realism (ESR) and Ontic Structural Realism (OSR): Ontic Structural Realism (OSR) is a metaphysical stance positing that structures are ontologically fundamental, meaning everything that exists is dependent on structures which themselves are independent. Epistemic Structural Realism (ESR), on the other hand, is an epistemic stance that claims our best scientific theories provide knowledge about the unobservable structure of the world, focusing on knowledge about structures rather than entities, processes, or events.
\end{enumerate}




% \bibliographystyle{gbt7714-numerical}
% % \bibliographystyle{7714-author-year}
% \bibliographystyle{ieeetr}
% \bibliography{bibl}

\end{document}


