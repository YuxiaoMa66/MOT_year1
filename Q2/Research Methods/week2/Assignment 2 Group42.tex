% \special{dvipdfmx:config z 0}
\documentclass[UTF8,a4paper,AutoFakeBold,AutoFakeSlant]{article}
\usepackage[a4paper,left=2.8cm,right=2.6cm,top=3.7cm,bottom=3.5cm]{geometry}
\usepackage{ctex}
% \usepackage{xeCJK}
\usepackage{graphicx}
\usepackage{pythonhighlight}
\usepackage[mathscr]{eucal}
\usepackage{mathrsfs}
\usepackage{booktabs}
\usepackage{capt-of} 
\usepackage{hyperref} 
\usepackage{abstract}
\usepackage{amsmath}
\usepackage{listings}
\usepackage{color}
\usepackage{caption}
\usepackage{subfigure}
\usepackage{enumerate}
\usepackage{amsfonts} 
\usepackage{CJK,CJKnumb}
\usepackage{float}
% \usepackage{gbt7714}
\usepackage{framed}
\usepackage{multirow}
\usepackage{animate}
\usepackage[framemethod=tikz]{mdframed}

\newcommand{\tnewroman}{\fontspec{Times New Roman}}
\newcommand{\song}{\CJKfamily{song}}    % 宋体   (Windows自带simsun.ttf)
\newcommand{\fs}{\CJKfamily{fs}}        % 仿宋体 (Windows自带simfs.ttf)
\newcommand{\kai}{\CJKfamily{kai}}      % 楷体   (Windows自带simkai.ttf)
\newcommand{\hei}{\CJKfamily{hei}}      % 黑体   (Windows自带simhei.ttf)
\newcommand{\li}{\CJKfamily{li}}        % 隶书   (Windows自带simli.ttf) 
\newcommand{\ssong}{\CJKfamily{STSong}}
% \newfontfamily{\heiti}{SimHei}

\xeCJKsetup{SlantFactor = 0.3}
% \xeCJKsetup{SlantFactor = -0.7}
\setCJKmainfont[BoldFont=SimHei, SlantedFont=KaiTi]{SimSun}



\usepackage{xcolor}  	%高亮使用的颜色
\definecolor{commentcolor}{RGB}{85,139,78}
\definecolor{stringcolor}{RGB}{206,145,108}
\definecolor{keywordcolor}{RGB}{34,34,250}
\definecolor{backcolor}{RGB}{220,220,220}

\usepackage{accsupp}	
\newcommand{\emptyaccsupp}[1]{\BeginAccSupp{ActualText={}}#1\EndAccSupp{}}






\title{\textbf{\textsf{{\textsf{Assignment 2}}}}} 
\author{\tnewroman Mihkel Kaalep ~\&~ Yuxiao Ma}
\date{}

% 去掉红框
\hypersetup{
colorlinks=true,
linkcolor=black
}

\begin{document}



\maketitle

\section{Research Question}

To what extent does the use of generative AI improve creativity of employees, and how does this relationship impact the overall innovation output of an organization?


\section{Core Constructs}

\begin{itemize}
    \item Generative AI adoption
\end{itemize}

Organization’s adoption of AI systems, for instance ChatGPT, to autonomously generate content, ideas, or solutions. A variable to measure this could be the number of prompts entered into chatGPT in a day.

\begin{itemize}
    \item Employee creativity
\end{itemize}

The level of novel and valuable ideas, solutions, or expressions generated by individuals within an organization.

\begin{itemize}
    \item Organizational innovative output
\end{itemize}

Company’s ability to develop and introduce high-impact products to the market in a timely manner. Variables to measure this could be the revenue from new products or time-to-market.



\section{Hypotheses}

\begin{itemize}
    \item Hypothesis 1
\end{itemize}

The use of generative AI in an organization positively influences the creativity of its employees.

(Denote: $R$ for correlation between the use of generative AI and creativity of employees)
\begin{equation}
    H_0: R>0; ~~H_a: R\leq 0
\end{equation}

\begin{itemize}
    \item Hypothesis 2
\end{itemize}

Higher creativity levels lead to better innovation outcomes.

\begin{itemize}
    \item Hypothesis 3
\end{itemize}

The use of generative AI reduces the risk in R\&D innovation phase process.

(Denote: $r_0$ for the risk in R\&D innovation phase process without using generative AI
$r_1$ for risk with generative AI)
\begin{equation}
    H_0: r_0-r_1>0; ~~H_a: r_0-r_1 \leq 0
\end{equation}


\section{Unit of Analysis}

\begin{enumerate}
    \item Employees: Individuals in AI-utilizing organizations, assessed for creativity levels.
    \item Organizations: Companies using generative AI, analyzed for innovation output metrics.
\end{enumerate}


\section{Research Strategy}

\begin{itemize}
    \item Survey Strategy
\end{itemize}

In order to test the internal validity of the hypotheses, a survey approach is chosen. A questionnaire will be distributed to employees in various companies to minimize the effect of any confounding factors and to measure the variables relating to the three key constructs. Questions need to include a timeframe of the introduction of the generative AI methods to make sure that the use of generative AI precedes the innovation outcomes. This is important for establishing causality. Further statistical methods will be used to find out if the variables co-vary. 





% \section{Improved research strategy \& Reflection}

% In order to enhance internal validity and rigorously test the hypotheses, an experimental research strategy is chosen. Unlike a survey, an experiment allows for the establishment of a causal relationship between variables, thereby strengthening internal validity.

% By manipulating the independent variable (e.g., introducing a specific generative AI training program) and observing its impact on dependent variables (e.g., employee creativity and organizational innovation output), experimental designs provide a more robust control over potential confounding factors. This control is essential for establishing a cause-and-effect relationship, a critical factor in assessing internal validity.

% The shift from a survey approach to an experimental research strategy was motivated by the desire to enhance internal validity in testing hypotheses. While the survey approach is effective in capturing associations among variables, it falls short in establishing causal relationships due to potential confounding factors.



% create a appendix
\newpage
\appendix
\section{Appendix: Improved research strategy \& Reflection}

\begin{itemize}
    \item Prompt1:
\end{itemize}

Define a research strategy. Make a choice between survey, experiment or case study. Motivate your choice, using terms of internal validity. (200 words) use the following as a base: In order to test the internal validity of the hypotheses, a survey approach is chosen. A questionnaire will be distributed to employees to measure the variables relating to the three key constructs. Further statistical methods will be used to analyze and find correlations.

\begin{itemize}
    \item Response1:
\end{itemize}

In order to enhance internal validity and rigorously test the hypotheses, an experimental research strategy is chosen. Unlike a survey, an experiment allows for the establishment of a causal relationship between variables, thereby strengthening internal validity.

By manipulating the independent variable (e.g., introducing a specific generative AI training program) and observing its impact on dependent variables (e.g., employee creativity and organizational innovation output), experimental designs provide a more robust control over potential confounding factors. This control is essential for establishing a cause-and-effect relationship, a critical factor in assessing internal validity.

Moreover, random assignment of participants to different experimental conditions helps distribute extraneous variables evenly, reducing the risk of alternative explanations for observed effects. The experimental design's high internal validity ensures that any changes in the dependent variables can be confidently attributed to the manipulation of the independent variable, providing a more robust foundation for drawing causal inferences.

Statistical analyses, such as ANOVA or regression, can still be employed to quantify the relationship and test the hypotheses within the experimental framework, ensuring a comprehensive examination of the research questions while prioritizing the internal validity of the study.

\begin{itemize}
    \item Prompt2:
\end{itemize}

Why we should shift from a survey to a experiment?

\begin{itemize}
    \item Response2:
\end{itemize}

The shift from a survey approach to an experimental research strategy was motivated by the desire to enhance internal validity in testing hypotheses. While the survey approach is effective in capturing associations among variables, it falls short in establishing causal relationships due to a lack of precedence in time.



% \bibliographystyle{gbt7714-numerical}
% % \bibliographystyle{7714-author-year}
% \bibliographystyle{ieeetr}
% \bibliography{bibl}

\end{document}