% \special{dvipdfmx:config z 0}
\documentclass[UTF8,a4paper,AutoFakeBold,AutoFakeSlant]{article}
\usepackage[a4paper,left=2.8cm,right=2.6cm,top=3.7cm,bottom=3.5cm]{geometry}
\usepackage{ctex}
% \usepackage{xeCJK}
\usepackage{graphicx}
\usepackage{pythonhighlight}
\usepackage[mathscr]{eucal}
\usepackage{mathrsfs}
\usepackage{booktabs}
\usepackage{capt-of} 
\usepackage{hyperref} 
\usepackage{abstract}
\usepackage{amsmath}
\usepackage{listings}
\usepackage{color}
\usepackage{caption}
\usepackage{subfigure}
\usepackage{enumerate}
\usepackage{amsfonts} 
\usepackage{CJK,CJKnumb}
\usepackage{float}
% \usepackage{gbt7714}
\usepackage{framed}
\usepackage{multirow}
\usepackage{animate}
\usepackage[framemethod=tikz]{mdframed}

\newcommand{\tnewroman}{\fontspec{Times New Roman}}
\newcommand{\song}{\CJKfamily{song}}    % 宋体   (Windows自带simsun.ttf)
\newcommand{\fs}{\CJKfamily{fs}}        % 仿宋体 (Windows自带simfs.ttf)
\newcommand{\kai}{\CJKfamily{kai}}      % 楷体   (Windows自带simkai.ttf)
\newcommand{\hei}{\CJKfamily{hei}}      % 黑体   (Windows自带simhei.ttf)
\newcommand{\li}{\CJKfamily{li}}        % 隶书   (Windows自带simli.ttf) 
\newcommand{\ssong}{\CJKfamily{STSong}}
% \newfontfamily{\heiti}{SimHei}

\xeCJKsetup{SlantFactor = 0.3}
% \xeCJKsetup{SlantFactor = -0.7}
\setCJKmainfont[BoldFont=SimHei, SlantedFont=KaiTi]{SimSun}



\usepackage{xcolor}  	%高亮使用的颜色
\definecolor{commentcolor}{RGB}{85,139,78}
\definecolor{stringcolor}{RGB}{206,145,108}
\definecolor{keywordcolor}{RGB}{34,34,250}
\definecolor{backcolor}{RGB}{220,220,220}

\usepackage{accsupp}	
\newcommand{\emptyaccsupp}[1]{\BeginAccSupp{ActualText={}}#1\EndAccSupp{}}






\title{\textbf{\textsf{{\textsf{Assignment 3}}}}} 
\author{\tnewroman Mihkel Kaalep ~\&~ Yuxiao Ma}
\date{}

% 去掉红框
\hypersetup{
colorlinks=true,
linkcolor=black
}

\begin{document}



\maketitle

\textbf{Research question}:

To what extent does the use of generative AI improve creativity of employees and the overall innovation output of an organization?

\section{Core constructs}

\begin{enumerate}
    \item \textbf{Generative AI Adoption} - The incorporation of large language models into business systems. An indirect metric for measuring this integration could be the average daily number of prompts submitted to ChatGPT. (adapted version to be non-circular)
    \item \textbf{Employee creativity} - The level of novel and valuable ideas, solutions, or expressions generated by individuals within an organization.
    \item \textbf{Organizational innovative output} - Company’s ability to develop and introduce high-impact products to the market in a timely manner. Variables to measure this could be the revenue from new products or time-to-market.
\end{enumerate}


\section{Survey questions}

\subsection{Generative AI Adoption}

\begin{enumerate}
    \item How strongly do you agree or disagree with the following statement: ‘Generative AI possesses significant capabilities to enhance creative processes.’
    \begin{itemize}
        \item strongly agree, agree, neutral, disagree, strongly disagree
    \end{itemize}
    \item On a scale of 1 to 7, how confident are you in using generative AI models like ChatGPT for complex tasks in your daily responsibilities?
    \begin{itemize}
        \item Totally untrustworthy 1 - 2 - 3 - 4 - 5 - 6 - 7 Totally Trustworthy
    \end{itemize}
    \item  Please rate your usage of ChatGPT (or similar services) over the past 7 days on the scale between the following extremes:
    \begin{itemize}
        \item Very Infrequently 1 - 2 - 3 - 4 - 5 - 6 - 7 Very Frequently
    \end{itemize}
\end{enumerate}


\subsection{Employee creativity}

\begin{enumerate}
    \item Estimate the percentage of ideas that successfully transition from the ideation stage to execution (from 0\% to 100\%).
    \item According to your opinion, how strongly do you agree or disagree with the statement: ‘ The use of generative AI in our organization has significantly enhanced the creativity of our employees.’
    \begin{itemize}
        \item strongly agree, agree, neutral, disagree, strongly disagree
    \end{itemize}
    \item Rank the following factors in order of their importance in enhancing employee creativity in your organization, where 1st is the most important and 5th is the least important.
    \begin{itemize}
        \item Implementation of generative AI tools
        \item Training and Development Programs
        \item Organizational Culture
        \item Employee Autonomy and Flexibility
        \item Access to Diverse Information and Resources
    \end{itemize}
\end{enumerate}


\subsection{Organizational innovative output}

\begin{enumerate}
    \item On a scale from 'Significantly Slower' to 'Significantly Faster', please indicate your perception of how the implementation of generative AI has affected the speed of innovation in your field.
    \begin{itemize}
        \item Significantly Slower 1 - 2 - 3 - 4 - 5 - 6 - 7 Significantly Faster
    \end{itemize}
    \item What type of innovations has your organization primarily focused on since the integration of generative AI?
    \begin{itemize}
        \item Product Innovation
        \item Process Innovation
        \item Business Model Innovation
        \item Service Innovation
        \item None of the above
    \end{itemize}
    \item By approximately what percentage has your organization's revenue from new products increased or decreased since the implementation of generative AI.
    \begin{itemize}
        \item Decreased more than 20\% (-$\infty$, -20\%)
        \item Decreased 10-20\% [-20\%,-10\%)
        \item Decreased 0-10\% [-10\%,0)
        \item Increased 0-10\%[0, 10\%)
        \item Increased 10-20\% [10\%, 20\%)
        \item Increased more than 20\% [20\%, +$\infty$)
    \end{itemize}
\end{enumerate}



\section{Population of the study}

Chief executives in the Netherlands


\section{Sampling approach}

\begin{itemize}
    \item \textbf{Sampling frame}: Netherlands chamber of commerce business register.
    \item \textbf{Sampling Design}: Probability Sampling. We have opted for a simple random sampling approach in our sampling design. This decision stems from our broad interest in studying the impact of generative AI across various business sectors without specific industry focus. The reason behind choosing simple random sampling is to ensure an equal and unbiased representation of businesses from all sectors in our study.
\end{itemize}





% \bibliographystyle{gbt7714-numerical}
% % \bibliographystyle{7714-author-year}
% \bibliographystyle{ieeetr}
% \bibliography{bibl}

\end{document}