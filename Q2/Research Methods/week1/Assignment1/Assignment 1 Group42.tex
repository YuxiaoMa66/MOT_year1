% \special{dvipdfmx:config z 0}
\documentclass[UTF8,a4paper,AutoFakeBold,AutoFakeSlant]{article}
\usepackage[a4paper,left=2.8cm,right=2.6cm,top=3.7cm,bottom=3.5cm]{geometry}
\usepackage{ctex}
% \usepackage{xeCJK}
\usepackage{graphicx}
\usepackage{pythonhighlight}
\usepackage[mathscr]{eucal}
\usepackage{mathrsfs}
\usepackage{booktabs}
\usepackage{capt-of} 
\usepackage{hyperref} 
\usepackage{abstract}
\usepackage{amsmath}
\usepackage{listings}
\usepackage{color}
\usepackage{caption}
\usepackage{subfigure}
\usepackage{enumerate}
\usepackage{amsfonts} 
\usepackage{CJK,CJKnumb}
\usepackage{float}
% \usepackage{gbt7714}
\usepackage{framed}
\usepackage{multirow}
\usepackage{animate}
\usepackage[framemethod=tikz]{mdframed}

\newcommand{\tnewroman}{\fontspec{Times New Roman}}
\newcommand{\song}{\CJKfamily{song}}    % 宋体   (Windows自带simsun.ttf)
\newcommand{\fs}{\CJKfamily{fs}}        % 仿宋体 (Windows自带simfs.ttf)
\newcommand{\kai}{\CJKfamily{kai}}      % 楷体   (Windows自带simkai.ttf)
\newcommand{\hei}{\CJKfamily{hei}}      % 黑体   (Windows自带simhei.ttf)
\newcommand{\li}{\CJKfamily{li}}        % 隶书   (Windows自带simli.ttf) 
\newcommand{\ssong}{\CJKfamily{STSong}}
% \newfontfamily{\heiti}{SimHei}

\xeCJKsetup{SlantFactor = 0.3}
% \xeCJKsetup{SlantFactor = -0.7}
\setCJKmainfont[BoldFont=SimHei, SlantedFont=KaiTi]{SimSun}



\usepackage{xcolor}  	%高亮使用的颜色
\definecolor{commentcolor}{RGB}{85,139,78}
\definecolor{stringcolor}{RGB}{206,145,108}
\definecolor{keywordcolor}{RGB}{34,34,250}
\definecolor{backcolor}{RGB}{220,220,220}

\usepackage{accsupp}	
\newcommand{\emptyaccsupp}[1]{\BeginAccSupp{ActualText={}}#1\EndAccSupp{}}






\title{\textbf{\textsf{{\textsf{Assignment 1}}}}} 
\author{\tnewroman Mihkel Kaalep ~\&~ Yuxiao Ma}
\date{}

% 去掉红框
\hypersetup{
colorlinks=true,
linkcolor=black
}

\begin{document}



\maketitle


\section{Problem \& Context}

Generative AI has emerged as a disruptive technology that can be used for generating ideas and improving workflows. We’ve used generative AI in our previous work for solving different tasks, but the problem was that the output was not always accurate. In-depth case studies on firms that have successfully implemented generative AI in their innovation processes would be valuable to find out what kind of an impact it has had on their innovation outcomes and how have they managed the risks associated with AI use. 

\section{Further Reasoning}

The choice is rooted in observation and a strong curiosity about generative AI’s real-world impact on companies’ innovation. This area emerged from witnessing the market trend that companies embrace AI or even train and publish their own type of AI. They struggle to harness AI effectively and there’s a clear desire to improve the current status quo, where maybe quantifiable benefits of AI in impacting innovation remain under explored. Therefore, the demand for learning and researching the generative AI’s role in innovative and creative processes is vital and could serve as a basis for developing a comprehensive framework for AI implementation.


\section{Research Objective}

The goal of this research is to explore the impact of generative AI on creativity and innovation outcomes.


\section{Main Research Question}

What is the frequency and nature of innovative outputs attributed to the use of generative AI in companies?



% \bibliographystyle{gbt7714-numerical}
% % \bibliographystyle{7714-author-year}
% \bibliographystyle{ieeetr}
% \bibliography{bibl}

\end{document}