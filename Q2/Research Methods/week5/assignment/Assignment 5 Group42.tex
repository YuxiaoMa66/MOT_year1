% \special{dvipdfmx:config z 0}
\documentclass[UTF8,a4paper,AutoFakeBold,AutoFakeSlant]{article}
\usepackage[a4paper,left=2.8cm,right=2.6cm,top=3.7cm,bottom=3.5cm]{geometry}
\usepackage{ctex}
% \usepackage{xeCJK}
\usepackage{graphicx}
\usepackage{pythonhighlight}
\usepackage[mathscr]{eucal}
\usepackage{mathrsfs}
\usepackage{booktabs}
\usepackage{capt-of} 
\usepackage{hyperref} 
\usepackage{abstract}
\usepackage{amsmath}
\usepackage{listings}
\usepackage{color}
\usepackage{caption}
\usepackage{subfigure}
\usepackage{enumerate}
\usepackage{amsfonts} 
\usepackage{CJK,CJKnumb}
\usepackage{float}
% \usepackage{gbt7714}
\usepackage{framed}
\usepackage{multirow}
\usepackage{animate}
\usepackage[framemethod=tikz]{mdframed}

\newcommand{\tnewroman}{\fontspec{Times New Roman}}
\newcommand{\song}{\CJKfamily{song}}    % 宋体   (Windows自带simsun.ttf)
\newcommand{\fs}{\CJKfamily{fs}}        % 仿宋体 (Windows自带simfs.ttf)
\newcommand{\kai}{\CJKfamily{kai}}      % 楷体   (Windows自带simkai.ttf)
\newcommand{\hei}{\CJKfamily{hei}}      % 黑体   (Windows自带simhei.ttf)
\newcommand{\li}{\CJKfamily{li}}        % 隶书   (Windows自带simli.ttf) 
\newcommand{\ssong}{\CJKfamily{STSong}}
% \newfontfamily{\heiti}{SimHei}

\xeCJKsetup{SlantFactor = 0.3}
% \xeCJKsetup{SlantFactor = -0.7}
\setCJKmainfont[BoldFont=SimHei, SlantedFont=KaiTi]{SimSun}



\usepackage{xcolor}  	%高亮使用的颜色
\definecolor{commentcolor}{RGB}{85,139,78}
\definecolor{stringcolor}{RGB}{206,145,108}
\definecolor{keywordcolor}{RGB}{34,34,250}
\definecolor{backcolor}{RGB}{220,220,220}

\usepackage{accsupp}	
\newcommand{\emptyaccsupp}[1]{\BeginAccSupp{ActualText={}}#1\EndAccSupp{}}






\title{\textbf{\textsf{{\textsf{Assignment 5}}}}} 
\author{\tnewroman Mihkel Kaalep ~\&~ Yuxiao Ma}
\date{}

% 去掉红框
\hypersetup{
colorlinks=true,
linkcolor=black
}

\renewcommand{\refname}{References}

\begin{document}



\maketitle


\section{Research Question}

How does the implementation of generative AI tools influence employee creativity and en- hance the overall innovation output within organizations?


\section{Qualitative \& Quantitative Research}

\begin{enumerate}
    \item \textbf{Data Type}: Quantitative research primarily uses numerical data, while qualitative research focuses on textual or visual data. Our initial goal was to find the frequency and nature of the results of generative AI innovation in companies: the quantitative results. Moving to qualitative requires further investigation into how and why generative AI affects creativity and innovation, perhaps through interviews and case studies.
    \item \textbf{Approach to Understanding}: Quantitative research seeks to quantify variables and generalize results from a sample to the population, whereas qualitative research aims to gain a deeper understanding of specific contexts or groups. In our research, this would mean moving from quantifying the impact of generative AI to understanding the subjective experiences and perceptions of those working with it.
    \item \textbf{Research Design Flexibility}: Quantitative research often uses a fixed design, but qualitative research is more flexible, allowing for changes as the research progresses. Initially, we planned a structured survey or experiment. With a qualitative approach, we would opt for semi-structured interviews or participant observation that can evolve based on initial findings.
    \item \textbf{Role of the Researcher}: In quantitative research, the researcher remains distant and objective, but in qualitative research, the researcher often becomes immersed in the situation and may influence it. For our study, this would mean a shift from maintaining distance and analyzing data objectively to engaging more deeply with subjects, possibly affecting the study dynamics.
    \item \textbf{Outcome Focus}: Quantitative research emphasizes testing hypotheses or theories, while qualitative research centers on generating new theories and understanding meanings. Our original design tested specific hypotheses about generative AI's impact. A qualitative shift would mean focusing on generating new insights about how generative AI is integrated and perceived in creative processes.
\end{enumerate}


\section{Qualitative Data Examples}

\begin{enumerate}
    \item A quote from a VP of Engineering at Google Workspace, who explains how generative AI can help teams collaborate and develop new thinking: "Generative AI plays an outsized role in creating that culture by helping teams break loose from the demands of process to bring their visions to life. It also helps people create new perspectives, improve early-stage ideas, and maintain their creative flow."\cite{decremer2023}
    \item A transcript of an interview with a highly skilled worker who used generative AI to come up with a new product idea for a shoe company: "I used Duet AI to generate some names and images for the new product. It was really helpful to see different options and get inspired by them. I also liked how I could tweak the details and combine ideas to create something unique. I think generative AI made me more productive and creative."\cite{webpage}
    \item A news article that reports on the results of a study by MIT researchers, who found that generative AI can boost highly skilled workers' productivity by as much as 40\%: "The researchers found that workers who used generative AI performed significantly better than those who did not, both in terms of quantity and quality of output. They also reported higher levels of satisfaction, engagement, and confidence in their work. The researchers attributed these benefits to the ability of generative AI to augment human creativity and reduce cognitive load."\cite{anewstudy}
\end{enumerate}



\section{Research Questions (the topic of digital platforms)}

\begin{itemize}
    \item How do digital platforms in the data economy influence the competitive dynamics among businesses in various sectors?
    \item Why do certain digital platforms in the data economy gain more trust and engagement from users compared to others?
\end{itemize}


\section{Code}

Creativity Augmentation \&
Productivity Enhancement








% \bibliographystyle{gbt7714-numerical}
% % \bibliographystyle{7714-author-year}
\bibliographystyle{ieeetr}
\bibliography{bibl}


\end{document}