\documentclass[a4paper,10pt,UTF8]{scrartcl}
\usepackage{xltxtra}
\usepackage[]{xkeyval,polyglossia}
% \setmainlanguage[spelling=new]{german}
\usepackage[b5paper,top=2cm, bottom=2cm, left = 2.5cm, right=2cm]{geometry}
\usepackage[]{csquotes}
\usepackage[]{titlesec}
\usepackage[]{url}
\usepackage[]{paralist}
\usepackage[absolute]{textpos}
\usepackage[]{rotating}
\usepackage[]{scrpage2}
\usepackage[]{blindtext}
\usepackage{ctex}
\usepackage{graphicx}
\usepackage{hyperref} 
% \usepackage{abstract}
% \setlength{\baselineskip}{10pt}
\usepackage{indentfirst}
\usepackage{enumerate}
\usepackage{fontspec}


% \usepackage[a4paper,left=2.8cm,right=2.8cm,top=2.5cm,bottom=2.5cm]{geometry}
\usepackage{graphicx}
\usepackage{pythonhighlight}
\usepackage[mathscr]{eucal}
\usepackage{mathrsfs}
\usepackage{booktabs}
\usepackage{capt-of} 
\usepackage{hyperref} 
\usepackage{abstract}
\usepackage{amsmath}
\usepackage{listings}
\usepackage{color}
\usepackage{caption}
\usepackage{subfigure}
\usepackage{enumerate}
\usepackage{amsfonts} 
\usepackage{float}


\pagestyle{scrheadings}
\setheadsepline[\textwidth]{0.25pt}{}
\ohead{\headmark}
\ofoot[\pagemark]{\pagemark}
\cfoot{}
\chead{}

\ihead{Innovation Project Approaches in Biomedical and Smart Home Solutions}


\usepackage{xcolor}
\definecolor{msdarkblue}{RGB}{54,95,145}
\definecolor{msblue}{RGB}{79,129,189}
%\titleformat{\section}[form]{layout}{labellayout}{abstand}{davorcode}[danachcode]
\titleformat{\abstract}[hang]{\color{msdarkblue}\Large\sffamily\bfseries}{}{0pt}{\vspace*{-6pt}}
\titleformat{\section}[hang]{\color{msdarkblue}\Large\sffamily\bfseries}{}{0pt}{\vspace*{-6pt}}
\titleformat{\subsection}[hang]{\color{msblue}\large\sffamily\bfseries}{}{0pt}{\vspace*{-6pt}}
\titleformat{\subsubsection}[hang]{\color{msblue}\normalsize\sffamily\bfseries}{}{0pt}{\vspace*{-9pt}}
\setsansfont[ItalicFont={Cambria Italic},BoldFont={Cambria Bold},BoldItalicFont={Cambria Bold Italic}]{Cambria}
% \setmainfont[ItalicFont={Calibri Italic},BoldFont={Calibri Bold},BoldItalicFont={Calibri Bold Italic}]{Calibri}
\setmonofont[ItalicFont={Consolas Italic},BoldFont={Consolas Bold},BoldItalicFont={Consolas Bold Italic}]{Consolas}
\setlength{\parindent}{0pt}
\setlength{\parskip}{0.7em}
\usepackage{unicode-math}
\setmathfont{Cambria Math}
% \setlength{\parindent}{1em} 

% \setCJKmainfont{Microsoft YaHei}  % 微软雅黑
\newcommand{\tnewroman}{\fontspec{Times New Roman}}



\begin{document}
\setlength{\TPHorizModule}{1mm}
\setlength{\TPVertModule}{1mm}
\begin{titlepage}
~
\begin{textblock}{80}(-10,-10)
\begin{color}{msdarkblue}
\rule{3cm}{30cm}
\end{color}
\end{textblock}
% Logo white
\begin{textblock}{130}(30,30)
%\rule{2cm}{2cm}\\[2em]

{\noindent\Huge\tnewroman\textbf{Assignment 1}}\\



% include all 5 members names and student numbers here
{\noindent\Large\bfseries By: }
\newline
{\noindent\Large\bfseries\tnewroman Yuxiao Ma 5916305}
\newline
{\noindent\Large\bfseries\tnewroman Yuxiao Ma 5916305}
\newline
{\noindent\Large\bfseries\tnewroman Yuxiao Ma 5916305}
\newline
{\noindent\Large\bfseries\tnewroman Yuxiao Ma 5916305}
\newline
{\noindent\Large\bfseries\tnewroman Yuxiao Ma 5916305}

\end{textblock}
\begin{textblock}{20}(12,200)
\begin{rotate}{90}
{\huge\bfseries \textcolor{white}{Emerging and Breakthrough Technologies }}
\end{rotate}
\end{textblock}
\end{titlepage}
\tableofcontents
% \clearpage



\section{Introduction}

The development of innovation strongly depends on a company’s industry, product type(s), business goals, and market strategy. In this case analysis, two companies will be analyzed regarding their business goals and product types which will then be translated into an innovation approach. Both aim to revolutionize their respective markets but need tailored approaches to innovate effectively. Firstly, the business goals will be lined up with the effects of available innovation strategies. Secondly, the translation of these strategies into their respective business models will be described. 
% --------------------------------------------------------------
\section{Chapter 1: Case analysis}

BioSolution (BS) is a start-up company that focuses on the way in which bacterial DNA is diagnosed from samples derived from patients suffering from bacterial infections. The company wants to effectively minimize the size and costs of the procedures involved in diagnosing bacterial DNA. As such, their goal is to build a single apparatus capable of combining all steps of the complex DNA analysis which currently requires a specialized laboratory. The single DNA analyser device is aimed to be manufactured in a multitude and then distributed and monetized globally. In order to realize this operation, several engineers with different expertises in relevant scientific fields are hired to explore the possibilities and designs of such DNA analysers. 

The starting and end point of what BioSolution wants to achieve is clear. The idea originated from the fact that the process of bacterial DNA identification can be very time-consuming, while the application of processes in specialized laboratories is a very costly expense. This results in diagnoses being derived at a later time frame which can possibly be reduced and, consequently, enable faster diagnosis and treatment of the patients. By using the expertise from the engineers, the realization of these goals could become more obtainable. As development of the prototype progresses, the engineers could function as peer reviewers of the product to make sure no mistakes are made during the innovation project. Therefore, you can argue that the mission of BioSolution is to develop an accessible and simple product that combats the issue of delayed bacterial DNA diagnoses. Not only is this product a breakthrough innovation that is more efficient, but one that is cost-effective.

In contrast, Smart Home Solutions (SHS) is a growing enterprise that has already established its place in the market in the past decade. The company aims to diversify its product line in pursuit of capturing a different segment of the market. This will be done through the development of a cheaper alternative of their innovative computer which optimizes and regulates energy consumption in homes, targeting the mid segment of the market. Unlike BioSolution, SHS already possesses technical expertise, generated through the development of their core product, which can be leveraged in the development of the new one. This entails that the new product will be an incremental innovation, one that has similar functionalities, output, system requirements and elements to the original product but cheaper. Moreover, the company has gained a deep understanding of the market in its 10-years history. Therefore, we can assume that the company is familiar with general trends and can act upon them - in this case they have noticed the increase in demand for the SHS system, which means that there is no need for a costly profound research of the market to look for potential applications or users of their new product. Finally, SHS has established good connections with other members of their ecosystem such as energy providers, consumers and system providers, which means that they can use this infrastructure to innovate in parallel with their partners. In addition, we know that the innovation is of incremental nature so it will not disrupt the market and take other players out of business. Thus we can assume there will be no external opposition to innovation (coming from partners) or deterioration of the established relationships. Overall, we can conclude that the focus of innovation should be on a more cost-effective method to produce the cheaper version of the computer while maintaining the company’s share in the top segment of the market. 



% --------------------------------------------------------------
\section{Chapter 2: Approach descriptions}
BioSolution as a start-up is at a growing stage focusing on groundbreaking technology, while SHS is at a mature stage, optimizing existing technology. BioSolution requires a multi-disciplinary approach given its complex biomedical focus, whereas SHS can leverage existing relationships with suppliers, deep understanding of the technology and a well-known, established market. 

Adapting a hybrid method mixing Agile-Scrum (AS) with Stage-Gate (SG) elements to the BioSolution process towards its target of scaling up so that it can eventually be accessible to medical facilities world-wide, would be a good choice. Although BioSolution possesses the knowledge of how to conduct a DNA analysis, the challenge comes with fitting this knowledge into a device. Since it is envisioned as a breakthrough technology, the pool of existing solutions will be limited which calls for a research on feasible technical solutions. In addition, since the company is at start-up phase and yet to find its place on the market a profound marketing research is required. By the third stage of the Stage-Gate process both of these researches should be presented and carefully revised by the gate-keepers which is why we have opted to include elements from this innovation process. Borrowing from Agile-Scrum, the development process is iterative, which allows the team to adapt as new information becomes available. This is crucial due to the rapid development in bacterial DNA extraction and analysis (Arning \& Wilson, 2020)\cite{arning2020past}. Furthermore, Agile-Scrum allows for testing by the targeted users, being medical practitioners worldwide. For a new product, user feedback is crucial for development. Especially to discover requirements that are considered valuable to the targeted customers. In addition, Stage-Gate is for quality control and its gates serve as decision points for the continuation, modification, or termination of the project. Moreover, Stage-Gate allows for a strong market oriëntation to ensure a marketable product, reasoning the companies’ belief. Thus, BioSolution generates ideas and meanwhile they reduce the cost but still keep the standard and their end goal in mind. Using AS-SG hybrid methods allows this company to maintain the speed and adaptability of Agile while also ensuring that the project meets the criteria needed for biomedical technology. It is a tailor-made solution that respects both the complexity and urgency of their mission.

The platform development process is the most suitable approach for the SHS innovation project. Such a process consists of two developments; first, the platform is constructed (in this case, the computer) and later on, a variety of innovations based on this platform are developed (e.g. connections/extensions to the computer or system). This approach enables flexibility and variation of innovations. However, the platform itself needs to be stable, highly specified and modular in regard to other players in the industry, as the product facilitates the analysis of many appliances. More specifically, using a platform approach for Smart Home Solutions implies a product that acts as a connecting link between energy providers such as solar panels, heat pumps, wind-turbines, and geo-thermal installations, and a variety of smart domestic products such as washing machines, lighting, heating and air conditioning, electric boilers, and kitchen supplies. Another aspect of the platform approach apart from the rising product variety, is the reduced product cost for additional products using the same basis. Assuming product expansions use the same core functionality, the manufacturing and development cost can be reduced. Not only is the smart home computer a physical platform for all of these, usually externally developed, products. SHS additionally acts as a market platform by gathering information about active and passive energy usage of products, assuming they are legally allowed to. This IT aspect of the product allows for the ability to capture, analyze, and exchange data at a rapid pace, strengthening network effects. For example, heating and cooling systems can be examined which can be interesting for developers of heating and cooling appliances, home insulation companies, and perhaps government issued renovation projects. This results in another way of making money, allowing for a lower product price. 



% --------------------------------------------------------------
\section{Conclusion}

Innovation is not one-size-fits-all, as concluded with our case studies of BioSolution and Smart Home Solutions. Each strategy aligns with the respective company's goals, market needs, and technological complexities. BioSolution aims for disruptive innovation, requiring agility and strictness; therefore, the Agile-Scrum Stage-Gate hybrid approach is suggested as most beneficial. On the contrary, SHS is best suited for a platform development process that takes advantage of existing relationships with clients and facilitates data monetization. These cases highlight the importance of selecting an innovation approach that aligns with an organization’s specific objectives and context. Customizing or combining established methods can provide a tailored framework for managing the complexities and risks inherent in innovation.


By contrasting the two cases of BS and SHS, the importance of taking into account the business goals and the nature of innovation when choosing the most suitable innovation approach is demonstrated. For BS, the Agile-Scrum Stage-Gate hybrid approach is beneficial since it allows for flexibility and multi-disciplinary collaboration. On the contrary, SHS is best suited for a platform development process that takes advantage of existing relationships with clients and facilitates data monetization.







% 引用
\bibliographystyle{ieeetr}
\bibliography{bibl}

\end{document}