% \special{dvipdfmx:config z 0}
\documentclass[UTF8,a4paper,AutoFakeBold,AutoFakeSlant]{article}
\usepackage[a4paper,left=2.8cm,right=2.6cm,top=3.7cm,bottom=3.5cm]{geometry}
\usepackage{ctex}
% \usepackage{xeCJK}
\usepackage{graphicx}
\usepackage{pythonhighlight}
\usepackage[mathscr]{eucal}
\usepackage{mathrsfs}
\usepackage{booktabs}
\usepackage{capt-of} 
\usepackage{hyperref} 
\usepackage{abstract}
\usepackage{amsmath}
\usepackage{listings}
\usepackage{color}
\usepackage{caption}
\usepackage{subfigure}
\usepackage{enumerate}
\usepackage{amsfonts} 
\usepackage{CJK,CJKnumb}
\usepackage{float}
% \usepackage{gbt7714}
\usepackage{framed}
\usepackage{multirow}
\usepackage{animate}
\usepackage[framemethod=tikz]{mdframed}

\newcommand{\tnewroman}{\fontspec{Times New Roman}}
\newcommand{\song}{\CJKfamily{song}}    % 宋体   (Windows自带simsun.ttf)
\newcommand{\fs}{\CJKfamily{fs}}        % 仿宋体 (Windows自带simfs.ttf)
\newcommand{\kai}{\CJKfamily{kai}}      % 楷体   (Windows自带simkai.ttf)
\newcommand{\hei}{\CJKfamily{hei}}      % 黑体   (Windows自带simhei.ttf)
\newcommand{\li}{\CJKfamily{li}}        % 隶书   (Windows自带simli.ttf) 
\newcommand{\ssong}{\CJKfamily{STSong}}
% \newfontfamily{\heiti}{SimHei}

\xeCJKsetup{SlantFactor = 0.3}
% \xeCJKsetup{SlantFactor = -0.7}
\setCJKmainfont[BoldFont=SimHei, SlantedFont=KaiTi]{SimSun}



\usepackage{xcolor}  	%高亮使用的颜色
\definecolor{commentcolor}{RGB}{85,139,78}
\definecolor{stringcolor}{RGB}{206,145,108}
\definecolor{keywordcolor}{RGB}{34,34,250}
\definecolor{backcolor}{RGB}{220,220,220}

\usepackage{accsupp}	
\newcommand{\emptyaccsupp}[1]{\BeginAccSupp{ActualText={}}#1\EndAccSupp{}}






\title{\textbf{\textsf{{\textsf{Assignment: Teaching case Chapter 10}}}}} 
\author{\tnewroman Yuxiao Ma}
\date{}

% 去掉红框
\hypersetup{
colorlinks=true,
linkcolor=black
}

\begin{document}



\maketitle


\section{What are the advantages and disadvantages of the creative side of Google being run as a flexible and flat “technocracy”?}

Google's flexible and flat "technocracy" promotes innovation, allows for rapid decision-making, and emphasizes performance rather than hierarchy, allowing for dynamic product development. However, this can also lead to challenges related to scalability, management oversight, and inefficiencies due to a lack of structured hierarchy.


\section{How does Google’s culture influence the kind of employees it can attract and retain?}

Google's open, innovative culture attracts employees who value creativity, autonomy and the opportunity to work on cutting-edge projects. This culture supports retention by encouraging personal project time and fostering a sense of community and purpose among employees.


\section{What do you believe the challenges are in having very different structure and controls for Google’s creative side versus the other parts of the company?}

Balancing the informal, innovative culture of Google's creative side with the more structured, efficiency-oriented aspects of the company posed a challenge. Ensuring consistency and coherence across different parts of the organization without stifling creativity is a key challenge.


\section{Some analysts have argued that Google’s free-form structure and the 20 percent time to work on personal projects is possible only because Google’s prior success has created financial slack in the company. Do you agree with this? Would Google be able to continue this management style if it had closer competitors?}

The argument that Google's liberal structure and 20\% of personal project time is due to its financial success is valid to some extent. While financial slack has certainly facilitated this culture, the culture itself is what drives innovation and success. Google's ability to maintain this management style against its competitors depends on its ability to innovate and adapt while maintaining its core culture.













% \bibliographystyle{gbt7714-numerical}
% % \bibliographystyle{7714-author-year}
% \bibliographystyle{ieeetr}
% \bibliography{bibl}

\end{document}