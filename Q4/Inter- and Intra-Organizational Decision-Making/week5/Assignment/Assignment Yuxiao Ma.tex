% \special{dvipdfmx:config z 0}
\documentclass[UTF8,a4paper,AutoFakeBold,AutoFakeSlant]{article}
\usepackage[a4paper,left=2.8cm,right=2.6cm,top=3.7cm,bottom=3.5cm]{geometry}
\usepackage{ctex}
% \usepackage{xeCJK}
\usepackage{graphicx}
\usepackage{pythonhighlight}
\usepackage[mathscr]{eucal}
\usepackage{mathrsfs}
\usepackage{booktabs}
\usepackage{capt-of} 
\usepackage{hyperref} 
\usepackage{abstract}
\usepackage{amsmath}
\usepackage{listings}
\usepackage{color}
\usepackage{caption}
\usepackage{subfigure}
\usepackage{enumerate}
\usepackage{amsfonts} 
\usepackage{CJK,CJKnumb}
\usepackage{float}
% \usepackage{gbt7714}
\usepackage{framed}
\usepackage{multirow}
\usepackage{animate}
\usepackage[framemethod=tikz]{mdframed}

\newcommand{\tnewroman}{\fontspec{Times New Roman}}
\newcommand{\song}{\CJKfamily{song}}    % 宋体   (Windows自带simsun.ttf)
\newcommand{\fs}{\CJKfamily{fs}}        % 仿宋体 (Windows自带simfs.ttf)
\newcommand{\kai}{\CJKfamily{kai}}      % 楷体   (Windows自带simkai.ttf)
\newcommand{\hei}{\CJKfamily{hei}}      % 黑体   (Windows自带simhei.ttf)
\newcommand{\li}{\CJKfamily{li}}        % 隶书   (Windows自带simli.ttf) 
\newcommand{\ssong}{\CJKfamily{STSong}}
% \newfontfamily{\heiti}{SimHei}

\xeCJKsetup{SlantFactor = 0.3}
% \xeCJKsetup{SlantFactor = -0.7}
\setCJKmainfont[BoldFont=SimHei, SlantedFont=KaiTi]{SimSun}



\usepackage{xcolor}  	%高亮使用的颜色
\definecolor{commentcolor}{RGB}{85,139,78}
\definecolor{stringcolor}{RGB}{206,145,108}
\definecolor{keywordcolor}{RGB}{34,34,250}
\definecolor{backcolor}{RGB}{220,220,220}

\usepackage{accsupp}	
\newcommand{\emptyaccsupp}[1]{\BeginAccSupp{ActualText={}}#1\EndAccSupp{}}






\title{\textbf{\textsf{{\textsf{Assignment 5}}}}} 
\author{\tnewroman Yuxiao Ma}
\date{}

% 去掉红框
\hypersetup{
colorlinks=true,
linkcolor=black
}

\renewcommand{\refname}{References}

\begin{document}




\maketitle


% Individual Assignment based on group discussion.

Our group discussion went very smoothly because the topic of our discussion was very consistent with the task objective, which was to use artificial intelligence technology to improve coastal flood response. This made our discussion progress quickly, and we quickly assigned tasks and completed the work after confirming everyone's common understanding of using AI to improve coastal flooding.

We first discussed how artificial intelligence technology can be applied to coastal flood prediction and evacuation planning. Our discussion started with the specific question of "How to use AI to mitigate the impact of coastal flooding and promote evacuation planning." We agreed that the use of artificial intelligence can achieve more advanced flood predictions and play a key role in real-time evacuation planning, thereby mitigating the impact of these natural disasters on citizens, property and the environment. This not only supports the goals of the European Green Deal to achieve climate neutrality and enhance climate resilience, but also promotes the development of a circular economy and a net zero economy.

I proposed and explained the role of AI in digital transformation in the group document. By collecting and transforming real-life data, AI can improve flood management and evacuation planning and enhance the digital transformation of public services. This approach can not only improve the efficiency of emergency response, but also promote technological innovation in public safety and environmental management.

Afterwards, we discussed in detail the benefits to various stakeholders. We believe that the EU Flood Directive could adapt existing policies based on the prediction and response data provided by AI, which could bring new insights and focus to more effective coastal flood mitigation. In addition, insurance companies will benefit from this, as a lot of money can be saved through effective flood mitigation. For the general public, more effective flood mitigation measures will reduce the pressure of evacuation. Environmental protection organizations will also benefit, as the impact of coastal flooding on local flora and fauna will be reduced. Water management agencies will benefit from the application of new technologies in water level management and flood monitoring.

In further discussion, we emphasized the importance of data sharing and learning. The more data the AI ​​model uses, the more accurate the prediction results will be. This will encourage data sharing between European countries, so that we can learn together and make more accurate predictions. The establishment of a database of European wave data and coastal flood data can facilitate this learning process. In addition, countries can share strategies on how to use AI predictions for evacuation planning, helping evacuation planning agencies to better learn and draw lessons. All these models and data can also be shared with other regions around the world to help them learn about coastal flooding and evacuation planning.

Finally, in the process of making the PowerPoint, we chose a suitable template and Fahmi was responsible for creating and presenting it. We worked together to complete this work and the results were very satisfactory. Our team members demonstrated good teamwork spirit and efficient execution throughout the process, which laid a solid foundation for the completion of our mission.

% \bibliographystyle{gbt7714-numerical}
% % \bibliographystyle{7714-author-year}
% \bibliographystyle{ieeetr}
% \bibliography{bibl}


\end{document}
% \bibliography{bibl}


\end{document}