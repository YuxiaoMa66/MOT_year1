% \special{dvipdfmx:config z 0}
\documentclass[UTF8,a4paper,AutoFakeBold,AutoFakeSlant]{article}
\usepackage[a4paper,left=2.6cm,right=2.6cm,top=3.cm,bottom=2.8cm]{geometry}
\usepackage{ctex}
% \usepackage{xeCJK}
\usepackage{graphicx}
\usepackage{pythonhighlight}
\usepackage[mathscr]{eucal}
\usepackage{mathrsfs}
\usepackage{booktabs}
\usepackage{capt-of} 
\usepackage{hyperref} 
\usepackage{abstract}
\usepackage{amsmath}
\usepackage{listings}
\usepackage{color}
\usepackage{caption}
\usepackage{subfigure}
\usepackage{enumerate}
\usepackage{amsfonts} 
\usepackage{CJK,CJKnumb}
\usepackage{float}
% \usepackage{gbt7714}
\usepackage{framed}
\usepackage{multirow}
\usepackage{animate}
\usepackage[framemethod=tikz]{mdframed}

\newcommand{\tnewroman}{\fontspec{Times New Roman}}
\newcommand{\song}{\CJKfamily{song}}    % 宋体   (Windows自带simsun.ttf)
\newcommand{\fs}{\CJKfamily{fs}}        % 仿宋体 (Windows自带simfs.ttf)
\newcommand{\kai}{\CJKfamily{kai}}      % 楷体   (Windows自带simkai.ttf)
\newcommand{\hei}{\CJKfamily{hei}}      % 黑体   (Windows自带simhei.ttf)
\newcommand{\li}{\CJKfamily{li}}        % 隶书   (Windows自带simli.ttf) 
\newcommand{\ssong}{\CJKfamily{STSong}}
% \newfontfamily{\heiti}{SimHei}

\xeCJKsetup{SlantFactor = 0.3}
% \xeCJKsetup{SlantFactor = -0.7}
\setCJKmainfont[BoldFont=SimHei, SlantedFont=KaiTi]{SimSun}



\usepackage{xcolor}  	%高亮使用的颜色
\definecolor{commentcolor}{RGB}{85,139,78}
\definecolor{stringcolor}{RGB}{206,145,108}
\definecolor{keywordcolor}{RGB}{34,34,250}
\definecolor{backcolor}{RGB}{220,220,220}

\usepackage{accsupp}	
\newcommand{\emptyaccsupp}[1]{\BeginAccSupp{ActualText={}}#1\EndAccSupp{}}






\title{\vspace{-1.5cm}\textbf{\textsf{{\textsf{Individual Assessment}}}}} 
\author{\tnewroman Yuxiao Ma}
\date{}

% 去掉红框
\hypersetup{
colorlinks=true,
linkcolor=black
}

\renewcommand{\refname}{References}

\begin{document}




\maketitle


\section{Question 1}

\subsection{How You Work in Your Team}


Our team adheres to a very comprehensive team contract to guide our collaboration. From the writing and discussion process we agreed that our primary goal was to get good marks so we would achieve high standards in our project. To ensure fairness and efficiency, we divided tasks equally and emphasized the importance of diligence and quality.

We have discussions based on weekly tasks depending on their respective backgrounds and ideas to get everyone involved and capitalize on each member's strengths (and based on our last personality test from a theoretical standpoint mine seem to fit together quite well). According to our team contract: the facilitator is responsible for managing the discussion, making sure everyone's voice is heard, and making sure the team is on track. The Task Manager is responsible for documenting, managing time, and tracking deadlines. The Content Quality Manager reviews work for compliance with project standards, while the Writer supports content creation and team dynamics. But in reality we don't seem to have such clear boundaries when it comes to weekly task completion, and our discussions tend to yield good results and consensus and accomplish something. We continue to divide up the work and assist each other as needed when it comes time for a full REPORT later on.

We use WhatsApp for informal updates and meeting planning, and Microsoft Teams for file sharing and progress tracking. This dual communication approach ensured clarity and efficiency. And one of the nice things is that so far the 4 of us have never missed a discussion.


\subsection{How You Address Team Conflicts}

Actually, we have not had any major conflicts so far. But we have had some minor disagreements, such as when we were discussing the criteria for the Decision Matrix. 
Integration is our preferred method. This approach ensures that our decision-making is more comprehensive and inclusive, as it takes into account the expertise and perspectives of each member. For example, if we had a slight disagreement about what the criteria in the Decision Matrix were, but through discussion and negotiation we finally reached a logical and theoretical consensus. If the differences cannot be resolved through integration, we can vote for a compromise, electing a state that we all agree on, i.e., each party makes concessions in order to reach an agreement. Making a balance between our own needs and the needs of others.

In addition to internal discussions, we ask questions to the teacher or TA for advice. This way we have an external perspective to mediate the issue.

From the point of view of how satisfied we have been with our group work process so far, this approach has proven to be effective in quickly resolving conflicts and maintaining a positive team environment.



\section{Question 2}

\subsection{}

I choose the two concepts of positionality and intersectionality in week 1 and inclusive decision making in Week 3 -addressing the Wicked Problem of issues of the pandemic.

Positionality involves recognizing and reflecting on one's social status and privilege, while intersectionality examines how aspects of identity intersect to create unique experiences of discrimination or privilege \cite{collins2015intersectionality}.
I think these points are particularly important when analyzing the different impacts of the epidemic on different communities, while also ensuring that the decision-making process is fair and inclusive.

Taking the systematic planning for responding to the epidemic as an example, we must consider the needs of different communities, especially those that are most vulnerable to the epidemic, such as low-income communities (a logical level connecting the two concepts is that high-income communities generally do not prioritize and settle in such relatively high-risk locations). As a result, such risks are more common in low-income communities. In addition to the geographical location itself, these areas usually lack infrastructure and are more vulnerable to epidemic-related risks. To this end, it is possible to consider organizing communities to collect opinions, evaluate various regions, design measures such as improved allocation of medical resources, early warning systems, and vote on measures.

Inclusive decision-making emphasizes the participation of different stakeholders to ensure that decisions are fair and take into account multiple perspectives \cite{heijne2013effective}.
In the context of epidemic protection systems, this means creating participatory platforms that engage all relevant stakeholders, including local residents, health organizations, businesses and government agencies, in discussions and decision-making processes. For example, workshops held in various communities make it possible to collect feedback directly from those most affected by the epidemic. Through public consultations, trust and transparency can also be built. Through inclusive decision-making, projects are able to gain broader support and meet diverse needs, thereby achieving sustainable results.

However, in the process of responding to the epidemic, we also need to be vigilant about the problems that may arise from excessive government regulation. For example, during the COVID epidemic, the government of my country forced everyone to test nucleic acid every day. Once the nucleic acid code was not green, they were forced to be quarantined, and they were allowed to enter public places such as supermarkets based on the test results. This excessive regulation may deprive individuals of privacy and freedom. This practice may not only arouse public resistance, but also lead to increased distrust of the government. Therefore, in the decision-making process, we need to balance the relationship between public health and safety and personal privacy and freedom to ensure that policies are both effective and respect individual rights.

\subsection{}

In the process of responding to the epidemic, the needs and priorities of different stakeholders often conflict. For example, the contradiction between the local people who yearn for freedom and the relevant institutions is very obvious.

The people are more concerned about the freedom to resume normal life and oppose excessive regulatory measures such as mandatory isolation and daily testing, which are considered to deprive them of their privacy and basic freedom. They hope to have more personal choices and hope that the government will adopt more transparent and humane epidemic prevention policies.

On the other hand, relevant institutions including the government, testing agencies and vaccine manufacturers have different priorities. The government often hopes to reduce the losses caused by the spread of the virus and reduce management costs through mandatory management. Through mandatory measures such as daily testing and quarantine policies, the government believes that it can more effectively control the spread of the epidemic and protect public health. However, such mandatory measures have also caused serious violations of personal privacy and freedom, causing widespread dissatisfaction and opposition.

The root of this conflict is that the public wants to restore their freedom and normal life, while the relevant institutions are more inclined to achieve public health goals and economic interests through compulsory management. The public prioritizes the immediate restoration of their freedom of life, while the institutions are more concerned with controlling the epidemic and reducing costs through strict management measures.

\subsection{}

\begin{itemize}
    \item \textbf{Positive}: 
    
    Long-term economic development: Effective epidemic management measures can reduce the health risks and economic losses caused by the epidemic, thereby promoting long-term economic development. By reducing the risk of virus transmission and protecting life and property, social stability can be restored and maintained, and economic activities can gradually return to normal. In addition, the implementation of prevention and control measures will also create employment opportunities, especially in the fields of medical and public health, further stimulate the local economy and improve living standards.

Job creation: A large number of jobs will be created in the process of implementing and maintaining public health infrastructure. This will not only alleviate unemployment in the short term, but also promote long-term economic growth by increasing residents' income levels.
    \item \textbf{Negative}: 
    
    Infringement of personal privacy and freedom: Strict quarantine measures and daily mandatory testing can effectively control the epidemic, but they may also seriously infringe on personal privacy and freedom. This excessive regulation may lead to increased public distrust of the government and trigger widespread resistance. For example, the government's mandatory isolation of patients and requiring everyone to be tested daily, and only allowing entry to public places such as supermarkets based on test results, may be considered a serious infringement of personal freedom and privacy.

    Uneven distribution of resources: Large-scale public health infrastructure construction and maintenance requires a lot of resources, which may lead to the crowding out of other public service resources. For example, excessive focus on epidemic prevention and control may lead to neglect of other health issues, which in turn affects the overall public health level.
\end{itemize}



\section{Question 3}    

\subsection{}

The concepts of System 1 and System 2 thinking distinguish two modes of thinking: System 1 is automatic, effortless, associative, intuitive, and makes quick judgments based on intuition and past experience. Although this may lead to bias and errors in the long run, it can help us make quick decisions and generally accurate short-term predictions; System 2 is deliberate, effortful, conscious and logical, requiring more careful analysis and logical reasoning. This mode is crucial for solving complex problems and making decisions.

In our group work, when we discuss how AI can be used in our subject as a solution to our wicked problem, we usually use System 1 thinking to naturally generate ideas in our minds (the brainstorming process is also very similar to this). Then, by consciously using System 2 thinking, we can critically evaluate different aspects of the problem, consider different perspectives, and gain a more comprehensive understanding, thereby ensuring the reliability of our decisions.

\subsection{}

Understanding System 1 and System 2 thinking provided valuable insights for our group’s project on how the gaps in interests between stakeholders affected the decision-making process for an AI coastal flood protection measure in the Netherlands. Relying on System 1 may be intuitive in most situations. By learning both systems, it is essential to engage in more deliberate considerations, namely System 2.

In the decision-making process, it is important to slow down, apply System 2 thinking, and thoroughly analyze the different interests and perspectives of stakeholders. For example, we will take the time to evaluate the long-term impact of AI solutions on all parties involved, including local communities, environmental organizations, and government agencies. This approach will help us develop more balanced and sustainable solutions that address the complex dynamics of stakeholder interests. Employing both thinking systems simultaneously ensures a comprehensive understanding of the problem and leads to a more effective and fair decision-making process.






\appendix
\section{Appendix}

\subsection{Group report}

As is in the proposal. my group's research question is :
\begin{quote}
    How do interest gaps between stakeholders influence the decision-making process in
designing AI-enabled coastal flood mitigation in the Netherlands?
\end{quote}
So in this assignment, my answer to the question 2 is not similar to the group report.


\subsection{Individual submission of the in-class group discussions}

Based on our group discussion, we talked about the contract of group project. 
And  we agreed on the goals and process and requirements for completing our group assignments for this course.
Since two members of the group didn't show up, it was a consolidation of just the three of us (and I think our consolidation was awesome and detailed).

During this course discussion, I found learning to plan for the future of the group. Being able to agree on many of the details before starting can help in the subsequent development of the assignment and save time and effort. Moreover, it was another experience of harmonizing the different qualities of the group members.


My key impression is that it was harder than we thought to fit our idea of a wicked problem into its definition, but by framing the problem in terms of “environment” we managed to embed the topic of AI into our problem.

Subsequent discussions and further definitions were accomplished in a relatively seamless manner. Very good! :)



This week's process was a little tougher than the previous ones. We had some new ideas about our wicked problem and made attempts to discuss and modify it, which was subversive to our search for ACTORS. But we did our best to include the relevant parts and refine them through the 6 perspectives. Very tough, but we think we can!



This week when we tried to relate the information in biomimicry taxonomy and asknature.org to our topic, we felt it was a bit difficult because the core of our solution was based on AI assistance for coastal flooding problems, so it didn't seem particularly relevant. However, when we opened our eyes beyond direct correlation, we found the task very interesting! Great reefs, underwater shelters, etc. we tried to incorporate them into our project and got what we thought were great results haha!


Our group discussion went very smoothly because the topic of our discussion was very consistent with the task objective, which was to use artificial intelligence technology to improve coastal flood response. This made our discussion progress quickly, and we quickly assigned tasks and completed the work after confirming everyone's common understanding of using AI to improve coastal flooding.

We first discussed how artificial intelligence technology can be applied to coastal flood prediction and evacuation planning. Our discussion started with the specific question of "How to use AI to mitigate the impact of coastal flooding and promote evacuation planning." We agreed that the use of artificial intelligence can achieve more advanced flood predictions and play a key role in real-time evacuation planning, thereby mitigating the impact of these natural disasters on citizens, property and the environment. This not only supports the goals of the European Green Deal to achieve climate neutrality and enhance climate resilience, but also promotes the development of a circular economy and a net zero economy.

I proposed and explained the role of AI in digital transformation in the group document. By collecting and transforming real-life data, AI can improve flood management and evacuation planning and enhance the digital transformation of public services. This approach can not only improve the efficiency of emergency response, but also promote technological innovation in public safety and environmental management.

Afterwards, we discussed in detail the benefits to various stakeholders. We believe that the EU Flood Directive could adapt existing policies based on the prediction and response data provided by AI, which could bring new insights and focus to more effective coastal flood mitigation. In addition, insurance companies will benefit from this, as a lot of money can be saved through effective flood mitigation. For the general public, more effective flood mitigation measures will reduce the pressure of evacuation. Environmental protection organizations will also benefit, as the impact of coastal flooding on local flora and fauna will be reduced. Water management agencies will benefit from the application of new technologies in water level management and flood monitoring.

In further discussion, we emphasized the importance of data sharing and learning. The more data the AI ​​model uses, the more accurate the prediction results will be. This will encourage data sharing between European countries, so that we can learn together and make more accurate predictions. The establishment of a database of European wave data and coastal flood data can facilitate this learning process. In addition, countries can share strategies on how to use AI predictions for evacuation planning, helping evacuation planning agencies to better learn and draw lessons. All these models and data can also be shared with other regions around the world to help them learn about coastal flooding and evacuation planning.

Finally, in the process of making the PowerPoint, we chose a suitable template and Fahmi was responsible for creating and presenting it. We worked together to complete this work and the results were very satisfactory. Our team members demonstrated good teamwork spirit and efficient execution throughout the process, which laid a solid foundation for the completion of our mission.



% \bibliographystyle{gbt7714-numerical}
% % \bibliographystyle{7714-author-year}
\bibliographystyle{ieeetr}
\bibliography{bibl}


\end{document}
% \bibliography{bibl}


\end{document}ent