% \special{dvipdfmx:config z 0}
\documentclass[UTF8,a4paper,AutoFakeBold,AutoFakeSlant]{article}
\usepackage[a4paper,left=2.8cm,right=2.6cm,top=3.7cm,bottom=3.5cm]{geometry}
\usepackage{ctex}
% \usepackage{xeCJK}
\usepackage{graphicx}
\usepackage{pythonhighlight}
\usepackage[mathscr]{eucal}
\usepackage{mathrsfs}
\usepackage{booktabs}
\usepackage{capt-of} 
\usepackage{hyperref} 
\usepackage{abstract}
\usepackage{amsmath}
\usepackage{listings}
\usepackage{color}
\usepackage{caption}
\usepackage{subfigure}
\usepackage{enumerate}
\usepackage{amsfonts} 
\usepackage{CJK,CJKnumb}
\usepackage{float}
% \usepackage{gbt7714}
\usepackage{framed}
\usepackage{multirow}
\usepackage{animate}
\usepackage[framemethod=tikz]{mdframed}

\newcommand{\tnewroman}{\fontspec{Times New Roman}}
\newcommand{\song}{\CJKfamily{song}}    % 宋体   (Windows自带simsun.ttf)
\newcommand{\fs}{\CJKfamily{fs}}        % 仿宋体 (Windows自带simfs.ttf)
\newcommand{\kai}{\CJKfamily{kai}}      % 楷体   (Windows自带simkai.ttf)
\newcommand{\hei}{\CJKfamily{hei}}      % 黑体   (Windows自带simhei.ttf)
\newcommand{\li}{\CJKfamily{li}}        % 隶书   (Windows自带simli.ttf) 
\newcommand{\ssong}{\CJKfamily{STSong}}
% \newfontfamily{\heiti}{SimHei}

\xeCJKsetup{SlantFactor = 0.3}
% \xeCJKsetup{SlantFactor = -0.7}
\setCJKmainfont[BoldFont=SimHei, SlantedFont=KaiTi]{SimSun}



\usepackage{xcolor}  	%高亮使用的颜色
\definecolor{commentcolor}{RGB}{85,139,78}
\definecolor{stringcolor}{RGB}{206,145,108}
\definecolor{keywordcolor}{RGB}{34,34,250}
\definecolor{backcolor}{RGB}{220,220,220}

\usepackage{accsupp}	
\newcommand{\emptyaccsupp}[1]{\BeginAccSupp{ActualText={}}#1\EndAccSupp{}}






\title{\textbf{\textsf{{\textsf{Trial Exam I}}}}} 
\author{\tnewroman Yuxiao Ma}
\date{}

% 去掉红框
\hypersetup{
colorlinks=true,
linkcolor=black
}

\renewcommand{\refname}{References}

\begin{document}




\maketitle


\section{Question 1}

\subsection{}

\begin{itemize}
    \item Major concepts:
\end{itemize}

This article's major concepts contain the proximities framework, multiplexity in Social 
Network Analysis (SNA)

\begin{itemize}
    \item Research questions:
\end{itemize}

Relationship between personal and business relations in the 
biotechnology sector in four regions of Chile.
Identify how 
intra-cluster business relations emerge and to assess if the personal network structure in 
which the cluster is embedded determines the intra-cluster business network structure.

\begin{itemize}
    \item Resolution:
\end{itemize}

Personal relations between individuals 
will increase trust and reduce information asymmetries, increasing the probability that a 
formal business relation between firms will be generated.

By using a novel dataset of primary information about the relations between managers and owners of firms, it was possible to 
measure personal links directly and estimate their importance in formal collaborations 
between firms. The probability of creating a business relationship was estimated using 
a multilevel logistic model suitable for assessing the non-independence of observations 
present in social network data.

The results suggest a positive and highly significant relationship between personal and business relations in high-tech clusters in the context of 
an emerging economy.

If the policy objective is to increase collaborations and 
knowledge diffusion throughout a cluster, social activities to promote personal connections 
between the owners, managers and researchers in a cluster can be considered as a way of 
complementing traditional grants to enable new formal collaboration.


\subsection{}

The innovation system focus on the network of  institutions, innovative agents 
and the relationships between them. It emphasizes knowledge flows, institutional frameworks, and interactions among various actors. The aim is to improve overall innovation capacity and efficiency by optimizing interactions within the system.

The clustering approach focuses on geographically clustered companies, suppliers, and related institutions, emphasizing the collaboration and knowledge spillovers that come with geographic proximity. The clustering approach emphasizes specialization and competitive advantage among companies within a cluster, aiming to improve innovation ability and competitiveness in a specific region through close connections and synergies between firms.


\subsection{}

Personal proximity promotes collaboration between DBFS by building trust and reducing transaction costs. Previous personal relationships have been shown to significantly increase the likelihood of forming business relationships.

Social proximity Enhances trust and mutual understanding among DBFS through shared norms, values and professional networks. These social ties are consolidated through professional networks such as industry associations and conferences, facilitating knowledge exchange and cooperation.

Geographical proximity Further enhances cooperation between DBFS by promoting frequent face-to-face interaction and effective communication. The clustering effect allows these companies to enjoy innovation and synergies from local knowledge spillovers, shared resources, and supportive ecosystems.



\section{Question 2}

\subsection{}

\begin{enumerate}
    \item Doctorate graduates: It measures and reflects a high level of education and research competence. However, while this indicates a workforce equipped with advanced problem-solving skills, the economic benefits of this indicator may not be immediately apparent if there are not enough job opportunities to absorb these graduates.
    \item Venture capital expenditures: It measures venture capital investment as a percentage of GDP. High levels of venture capital spending indicate confidence in start-ups and innovative projects, helping to drive high-growth companies. However, this index is highly volatile and significantly affected by macroeconomic conditions, which does not necessarily guarantee the success of innovative projects.
    \item International scientific co-publications: It measures the number of scientific publications co-authored by researchers from different countries. It is an important indicator for assessing international cooperation and knowledge exchange. The number of international co-authored scientific publications per million people shows an increase in the quality and impact of research. However, this indicator focuses on academic results, rather than the commercialization or practical application of these research results, and therefore does not directly reflect the overall picture of innovation capacity.
    \item Employment in knowledge-intensive activities: It measures the share of total employment in highly knowledge-intensive industries. This indicator reflects the economy's focus on high-value, innovation-driven industries, indicating the potential for high productivity and technological advancement. However, a high share of employment in knowledge-intensive activities does not necessarily lead to broad economic benefits if other sectors are neglected.
    \item PCT patent applications: It measures the number of international patent applications filed under the Patent Cooperation Treaty (PCT). This indicator reflects the level of innovation and technological development in a country, as well as its ability to protect intellectual property rights globally. However, the number of patent applications does not guarantee the commercial success of these innovations or their contribution to economic growth.
\end{enumerate}


\subsection{}

While the Comprehensive Innovation Index (SII) provides a concise overview of national innovation performance, facilitating international comparisons and trend analysis by policymakers and stakeholders, it has also faced criticism. Edquist et al. (2018) point out that by aggregating multiple indicators into a single score, SII may mask important differences between individual indicators, leading to an oversimplification and potentially misleading view of innovation systems. The composite score may not reflect the complexity and diversity of innovative activities, thus affecting targeted policy formulation.

\subsection{}

Based on Marxt and Brunner's analysis, the Swiss innovation system has strong R\&D capabilities and a highly developed education system, especially in terms of PhD graduates and lifelong learning, which is also verified by the European Innovation Scoreboard. Switzerland has excelled in international scientific collaboration and highly cited publications, demonstrating strong research capabilities and global reach. In addition, Switzerland performs well in exports of high-tech goods and knowledge-intensive services, indicating that its innovations are competitive in the global market.

However, there are some weaknesses in the Swiss innovation system. Despite strong R\&D capabilities, government support for corporate R\&D is weak, which may limit the private sector's incentive to innovate. Innovation cooperation among smes also needs to be strengthened, indicating that there is room for improvement in networking and collaboration among enterprises. In addition, broadband penetration is relatively low, which could affect the development of digital transformation and connectivity. Addressing these shortcomings will help improve Switzerland's innovation system across the board.

\subsection{}

To improve the Swiss innovation level, the following strategies can be considered:

Increase government support for corporate research and development and encourage private sector innovation.

Strengthen the collaboration between industry and academia to enhance the commercialization of research results.

Attract international talent while investing in lifelong learning programs to maintain a highly skilled workforce.



% \bibliographystyle{gbt7714-numerical}
% % \bibliographystyle{7714-author-year}
% \bibliographystyle{ieeetr}
% \bibliography{bibl}


\end{document}
% \bibliography{bibl}


\end{document}